\chapter{Packages}
\label{chap:packages}

The \texttt{ntnuthesis} is built upon the standard \LaTeX\
\texttt{report} class. All commands from the \texttt{report} class can
be used, with the two exceptions of \verb+\subsubsection+ and
\verb+\paragraph+. This is because there should only be three
levels of headings according to the guidelines. 
It has been placed in a folder called \texttt{ntnuthesis} so that it does not
clutter your work.  You should not change anything in \texttt{ntnuthesis}. If you need to change 
anything you should make a pull request on the github repository for this thesis at
\url{https://github.com/COPCSE-NTNU/master-theses-NTNU}

\section{Packages Used by ntnuthesis}
\label{sec:packages}

In addition to the \texttt{report} document class,
\texttt{ntnuthesis} makes direct use of the following packages
that must hence be present:
\begin{description}
	\item[geometry:] used for setting the sizes of the margins and
  	headers.
	\item[fontenc:] used with option \texttt{T1} for forcing the Cork font
  	encoding (necessary for the Charter font).
	\item[charter:] load Charter as the default font.
	\item[euler:] load the Euler math fonts.
	\item[bable:] for language handling.
\end{description}

\section{Other Relevant Packages}
\label{sec:otherpackages}

The author of a thesis might want to use a bunch of different packages
to those described in Section~\ref{sec:packages} in order to have all features needed for their document. 
In particular, it is advised to use the following:
\begin{description}
	\item[inputenc:] to allow \LaTeX\ to use more than 7-bit ASCII for its
	  input. Most often, the option \texttt{latin1} will do.
	\item[babel:] to load language specific strings. Reasonable options
	  include \texttt{british}, 
		\texttt{american}, \texttt{norsk} and
	  \texttt{nynorsk}.
	\item[graphicx:] to include graphics.
	\item[hyperref:] this is a very nice package that makes cross links in
	  pdf documents. Use with option \texttt{dvips} or \texttt{pdftex}
	  in accordance with the driver that you use. Unfortunately, hyperref
	  is not completely bugfree\dots
\end{description}

We have web pages as well~\cite{NTNU:Website}, and now games like Halo~\cite{Halo}.